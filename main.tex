\documentclass[12pt,letterpaper]{book}
\usepackage[utf8]{inputenc}

% 1. The latex book format has pages too short
\addtolength{\topmargin}{-0.70in}
\addtolength{\textheight}{2.00in}
% 2. If you want to scroll well, you need to center the text:
\setlength{\oddsidemargin}{3.25in}
\addtolength{\oddsidemargin}{-0.5\textwidth}
\setlength{\evensidemargin}{\oddsidemargin}
% 3. Latex is broken
\renewcommand{\baselinestretch}{1.06}
\sloppy\sloppypar\raggedbottom\frenchspacing

\title{\textbf{The Practice of Astrophysics}}
\author{anonymous}
\date{April 2021}

\begin{document}

\maketitle

\cleardoublepage
\tableofcontents

\chapter*{Preface}
\addcontentsline{toc}{chapter}{Preface}

For the last few decades there has been a conversation going on in the astrophysics community about the disconnect between what we teach in the standard undergraduate and graduate core, and what we use in our day-to-day work as researchers.
Visualization! That's never taught to you, and many figures in many papers are hard to read.
Teaching! We are hired to teach but don't (without extra effort) obtain any training in education.
Statistics! We learn the folk activities of our advisors and supervisors, while doing research, and often wrongly.
These conversations lead to enthusiastic calls to revise the curriculum and teach the things that an astrophysicist \emph{really needs to know}. And yet.

And yet: What has changed? In my institution and many others, the core requirements for the PhD (prior to admission and within the program) have barely changed in these same few decades.

People (especially parents of physics-oriented students) often ask me what is the most important thing to know or learn if you want to become a scientist.
I can surprise them by saying ``reading and writing''.
Science is the scientific literature.
When we say that ``we know the age of the Universe to better than one percent'' we mean that there is a published (and probably refereed) scientific paper that explains and justifies this point.
We produce writing.
Astrophysicists are writers.
We are also designers, coders, builders, data-processors, statisticians, visualizers, mentors, teachers, advisors, speakers, proposers, and referees.
But even within these activities many of them connect to writing; these activities are often in the service of writing and publishing papers, which are the long-lasting products of our research.

There is another category of long-lasting product, though, and that is \emph{people}:
We educate, train, mentor, and launch people into careers in academia, in schools, in national labs, in technology companies, in policy positions, and in government.
This is probably the most important part of what we do, since it combines all the challenges of humanity with ethical and moral complexities, perhaps even more extreme than in other areas because \emph{astronomy has no point in itself}.
We only do astronomy because people want to do astronomy, and want to have astronomy done.
If people lost interest in this, we would have no role in society.
There is no urgent human need for astronomy, beyond the urgent human need to understand everything that can be understood.

\part{Reading and writing}%

\chapter{How to read a scientific paper}

\chapter{Writing abstracts and titles}

\chapter{Writing scientific papers}

\chapter{Grant proposals}

% look at an altac_naumes tweet from 2021-04-13 about grant reviews: "the first sentence(s) of a proposal should clearly state 1) what you plan to do; 2) how you plan to do it; and 3) why the funder should care. The introduction is not a lit review.

\chapter{White papers and reports}

\part{Teaching}

\chapter{Syllabus}

\chapter{Assignments and evaluation}

\chapter{Classroom practices}

\part{Scientific collaboration}

\chapter{Project design and design thinking}

\chapter{Authorship rules}

\chapter{Collaboration policies}

\chapter{Open science}

\part{Meetings}

\chapter{Giving a seminar}

\chapter{Attending a scientific meeting}

\chapter{Running a scientific meeting}

\chapter{Collaboration meetings}

\part{Data analysis}

\chapter{Model building}

\chapter{Probability}

\chapter{Information theory}

\chapter{Linear algebra}

\chapter{Optimization}

\chapter{Visualization}

\chapter{Data structures and databases}

\chapter{Data publication and release}

\part{Software and code}

\chapter{Good practice}

\chapter{Diagnosis and debugging}

\chapter{Open-source and community}

\part{Managing and supervising}

\chapter{Supervising student projects}

\chapter{Accepting supervision; managing up}

\chapter{Mentoring}

\chapter{Equity and Inclusion}

\chapter{Hiring}

\chapter{Making and maintaining budgets}

\chapter{Project management}

\part{Reviewing the work of others}

\chapter{Referee reports}

\chapter{Letters of recommendation}

\chapter{Proposal reviews}

\chapter{Tenure evaluations}

\end{document}
